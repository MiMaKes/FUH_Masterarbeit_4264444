%!TEX encoding = UTF-8 Unicode
\documentclass[12pt]{article} % 12pt-article hier, aber Abschlussarbeit dann 12pt-book
\usepackage[utf8]{inputenc}   %empfohlene Zeichenkodierung UTF-8
\usepackage[T1]{fontenc}      %empfohlene Fontkodierung
\usepackage{lmodern}          %besserer Font
\usepackage{microtype}        %bessere Zeichenabstände
\usepackage[german]{babel}    %deutsch

\usepackage[a4paper,margin=2.5cm]{geometry} %Seitenmaße
\parskip.5\baselineskip


\begin{document}

%% Titelabschnitt
\begin{center}
   \parskip1\baselineskip
   
   Exposé zur Abschlussarbeit am LG Kooperative Systeme der FernUniversität in Hagen
   
   ~
   
   {\LARGE\bfseries
      [Arbeitstitel]}

   \large
   [Name]
   
   [Matrikelnummer]
   
   [E-Mail]
   
   [Studiengang]
   
   [geplanter Zeitraum]
   
   Betreuer: [Betreuer am LG]
%}
\end{center}

%% Text
\section{Problemstellung}

\subsection{Motivation}
\begin{itemize}
\item Was ist das Problem, dass in der Arbeit gelöst werden soll?
\item Warum ist dies ein relevantes Problem für die Informatik (Motivation)?
\end{itemize}

\subsection{Aufgabenstellung}
\begin{itemize}
\item Was ist die konkrete Aufgabenstellung, das Ziel der Arbeit?
\end{itemize}

\subsection{Intendierte Ergebnisse}
\begin{itemize}
\item Was sind die intendierten Ergebnisse der Arbeit, die mit geeigneten Methoden
der Informatik erzielt werden sollen, und die das Problem lösen?
(z.\,B. Anforderungsdefinition, Spezifikation, Architektur, Algorithmen, Prototyp, Methode etc.)
\end{itemize}


\section{Aktueller Stand der Technik}

\begin{itemize}
\item Was gibt es an existierenden Lösungen/Ansätzen?
\item Welche Defizite haben diese, d.\,h., warum sind die nicht ausreichend
      zur Lösung des Problems?
\end{itemize}


\section{Lösungsidee}

\begin{itemize}
\item Welche Ideen haben Sie zur Lösung des Problems?
\item Mit welchen Schritten wird die Lösungsidee realisiert?
      \begin{itemize}
      \item Mit welchen Methoden der Informatik werden die intendierten
            Ergebnisse der Arbeit (s.\,o.) erreicht?
      \item Warum sind diese angemessen, um die Ergebnisse mit der notwendigen Qualität
            zu erreichen?
      \end{itemize}
      \item Wie wird im Rahmen der Arbeit geprüft (validiert), ob die Ergebnisse korrekt
            sind, d.\,h. das Problem tatsächlich lösen? Welche Unsicherheiten bleiben
            ggfs. als offene Fragen für Folgearbeiten bestehen?
\end{itemize}


\section{Vorläufige Gliederung}

\begin{itemize}
\item Gliederungsvorschlag (Kapitel und Abschnitte mit Stichworten für geplante Inhalte),
      der die Ergebnisse Ihres Vorgehens als Lösung der Aufgabenstellung
      (Problem, Ziel der Arbeit) nachvollziehbar und begründet darstellt.
\end{itemize}


\section{Vorläufiger Zeitplan}

\begin{itemize}
\item Abfolge der notwendigen Schritte inkl. Erstellung der jeweiligen Kapitel der Arbeit
\item Meilensteine von Anmeldung bis Abgabe der Arbeit
\end{itemize}


\section{Ausgangsliteratur}

\begin{itemize}
\item relevante Literaturquellen, hier im Exposé an den geeigneten Stellen referenziert
\item Zitierweise gemäß der Mustervorlage für Abschlussarbeiten
\end{itemize}


\section{Abschlussarbeit im Unternehmen}

Bei Arbeiten, die in einer Firma/Behörde/Institution durchgeführt werden sollen,
geben Sie bitte die dort betreuende Person an. Sie sollte idealerweise einen
Abschluss in Informatik haben und die Erstellung der Arbeit inhaltlich mitbetreuen.

\end{document}
