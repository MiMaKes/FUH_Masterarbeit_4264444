\usepackage[backend=biber,style=ieee,dashed=false]{biblatex} % Bibliothek
\bibliography{../bib.bib}
\usepackage{url}
\setcounter{biburllcpenalty}{7000}
\setcounter{biburlucpenalty}{8000}
\usepackage{makeidx} % Index
\makeindex

\usepackage{microtype} % optischer Randausgleich
\setlength{\parindent}{0pt}  % Absatz-Einrückung verhindern
\usepackage[pdftex]{color,graphicx} % Bilder und Farben
\usepackage{caption}
\usepackage{subcaption} % Subcaptions für Abbildungen
\usepackage{multirow} % mehrere Spalten als eine Tabellenzelle
\usepackage{makecell} % manuelle Zeilenumbrüche in Tabelle
\usepackage{rotating} % rotierte Abbildungen
\usepackage[usenames,dvipsnames,svgnames,table]{xcolor}
\usepackage{amsmath, amsthm, amsfonts, amssymb, wasysym} % Zeichen/Zusatzpakete laden
\usepackage{import}
\usepackage{scrhack} % Get rid of \float@addtolists warning
\usepackage{booktabs} % enhance table layout
\usepackage{float}%tabelle nicht am Anfang einer Seite
\usepackage{tabularx}
\usepackage{makecell}
\usepackage{ragged2e}
\newcolumntype{L}[1]{>{\raggedright\arraybackslash}p{#1}}
\newcolumntype{C}[1]{>{\centering\arraybackslash}p{#1}}
\newcolumntype{R}[1]{>{\raggedleft\arraybackslash}p{#1}}
\newcolumntype{J}[1]{>{\justifying\arraybackslash}p{#1}}
\newcommand*{\source}[1]{\par\raggedleft\scriptsize Quelle:~#1} %Abbildung Quellenangabe
\setcounter{secnumdepth}{2} %Gliederungtiefe sub, subsub, paragraph, subparagraph
\makeatletter
%\renewcommand\paragraph{\@startsection{paragraph}{4}{\z@}% linebreak after paragraph
%  {-3.25ex\@plus -1ex \@minus -.2ex}%
%  {1.5ex \@plus .2ex}%
%  {\normalfont\normalsize\bfseries}}
\makeatother

%Absatz und Überschrift Abstände einstellen
\RedeclareSectionCommand[
  beforeskip=-0\baselineskip,
  afterskip=.3\baselineskip]{chapter}
\RedeclareSectionCommand[
  beforeskip=-.8\baselineskip,
  afterskip=.25\baselineskip]{section}
\RedeclareSectionCommand[
  beforeskip=-.75\baselineskip,
  afterskip=.2\baselineskip]{subsection}
\RedeclareSectionCommand[
  beforeskip=-.5\baselineskip,
  afterskip=.15\baselineskip]{subsubsection}
\RedeclareSectionCommand[
  afterskip=1sp]{paragraph}






\usepackage{todonotes} %TODO Highlighting

\usepackage{listings} % Quelltexte
\newcounter{savedlinenumber} % Define a counter to store the line number
\newcommand{\savelinenumber}{\setcounter{savedlinenumber}{\value{lstnumber}}} % Define a macro to store the current line number
\newcommand{\restorelinenumber}{\setcounter{lstnumber}{\value{savedlinenumber}}} % Define a macro to restore the saved line number
\renewcommand{\lstlistlistingname}{Quelltextverzeichnis}
\renewcommand{\lstlistingname}{Quelltext}
% Farbdefinitionen für C-Syntaxhervorhebung
\definecolor{codegreen}{rgb}{0.42,0.6,0.33}
\definecolor{codegray}{rgb}{0.5,0.5,0.5}
\definecolor{codelightgreen}{rgb}{0.71,0.81,0.66}
\definecolor{codeblue}{rgb}{0.34,0.61,0.84}
\definecolor{codeorange}{rgb}{0.81,0.57,0.47}
\definecolor{backcolour}{rgb}{0.95,0.95,0.92}
% Konfiguration für C#
\lstdefinestyle{csharp}{
  language=[sharp]c,
  basicstyle=\ttfamily\scriptsize,
  backgroundcolor=\color{backcolour},
  commentstyle=\color{codegreen},
  keywordstyle=\color{codeblue},
  numberstyle=\tiny\color{codelightgreen},
  stringstyle=\color{codeorange},
  numbers=left,
  numberstyle=\tiny,
  breaklines=true,
  breakatwhitespace=true,
  showstringspaces=false,
  escapechar=|,
  tabsize=2,
  captionpos="b"
}
% Konfiguration für C++
\lstdefinestyle{cpp}{
  language=C++,
  basicstyle=\ttfamily\scriptsize,
  backgroundcolor=\color{backcolour},
  commentstyle=\color{codegreen},
  keywordstyle=\color{codeblue},
  numberstyle=\tiny\color{codelightgreen},
  stringstyle=\color{codeorange},
  numbers=left,
  numberstyle=\tiny,
  breaklines=true,
  breakatwhitespace=true,
  showstringspaces=false,
  escapechar=|,
  tabsize=2,
  captionpos="b"
}
% Konfiguration für textuelle Dokumention
\lstdefinelanguage{txtSpec}{
  morekeywords={actors, precondition, main_flow, alternative_flow, postcondition},
  morecomment=[l]{//},
  morecomment=[s]{/*}{*/},
  morestring=[b]",
  morestring=[b]**
}
\lstdefinestyle{txtSpecStyle}{
  language=txtSpec,
  basicstyle=\ttfamily\scriptsize,
  backgroundcolor=\color{backcolour},
  commentstyle=\color{codegreen},
  keywordstyle=\color{codeblue},
  numberstyle=\tiny\color{codelightgreen},
  stringstyle=\color{codeorange},
  numbers=none,
  breaklines=true,
  breakatwhitespace=true,
  showstringspaces=false,
  escapechar=|,
  tabsize=2,
  captionpos="b"
}
\def\ContinueLineNumber{\lstset{firstnumber=last}}
\def\StartLineAt#1{\lstset{firstnumber=#1}}
\let\numberLineAt\StartLineAt

\usepackage{algorithm} % Pseudocode
\usepackage{algpseudocode} % Pseudocode-Stil
\floatname{algorithm}{Algorithmus}
\renewcommand{\algorithmicrequire}{\textbf{Input:}}
\renewcommand{\algorithmicensure}{\textbf{Output:}}

\usepackage[german, hidelinks]{hyperref} % Verweise (im PDF Ausgabe Rahmen ausblenden)
\usepackage[german]{cleveref} % Verweise
\crefname{listing}{Quelltext}{Quelltexte}
\crefname{subsection}{Unterabschnitt}{Unterabschnitte}
\crefname{subsubsection}{Unterabschnitt}{Unterabschnitte}

\usepackage[printonlyused,nohyperlinks]{acronym} % Abkürzungsverzeichnis
\renewcommand*\aclabelfont[1]{\acsfont{#1}}

\usepackage{geometry} % Papierformat und Seitenabstände
\geometry{a4paper, left=2.8cm, right=2.8cm, top=3cm, bottom=4cm} % Papierformat und Seitenabstände
\linespread {1.2}\selectfont % Zeilenabstand

% Setzen Sie den Absatzeinzug
\setlength{\parindent}{1em}
\setlength{\parskip}{6pt}

\usepackage{csquotes} % um \enquote{text} zu nutzen
\usepackage{enumitem} % erweiterte Möglichkeiten für Aufzählungen

\newcommand{\imagedir}{../../Doc/Images}